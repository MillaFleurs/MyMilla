\documentclass[11pt]{article}
\usepackage[margin=1in]{geometry}
\usepackage[T1]{fontenc}
\usepackage{lmodern}
\usepackage{hyperref}
\setlength{\parskip}{0.7em}
\setlength{\parindent}{0pt}

\title{Milla Brain Install Guide}
\author{Project Docs}
\date{\today}

\begin{document}
\maketitle

\section*{Prerequisites}
\begin{itemize}
  \item OS: Ubuntu Linux (or similar). Works on macOS with analogous tools.
  \item Java (OpenJDK) and Clojure CLI (\texttt{clj}).
  \item SQLite available on the system.
  \item Ollama running locally at \texttt{http://localhost:11434} with a model name matching your config (e.g., \texttt{llama3.2}).
\end{itemize}

\section*{Install prerequisites on Ubuntu}
\begin{verbatim}
sudo apt-get update
sudo apt-get install -y curl gnupg ca-certificates \
    openjdk-21-jdk sqlite3

# Install Clojure CLI
curl -O https://download.clojure.org/install/linux-install-1.11.1.1435.sh
chmod +x linux-install-1.11.1.1435.sh
sudo ./linux-install-1.11.1.1435.sh

# Install Ollama (from upstream; see https://ollama.com)
curl -fsSL https://ollama.com/install.sh | sh
\end{verbatim}

\section*{Get the code}
\begin{verbatim}
cd ~/src   # or wherever you keep your projects
git clone https://github.com/MillaFleurs/MyMilla.git
cd MyMilla
# or if you prefer github
gh repo clone MillaFleurs/MyMilla
cd MyMilla
\end{verbatim}

\section*{Configure}
Runtime configuration is read from YAML in this order: \texttt{MILLA\_CONFIG} env var \textrightarrow{} \texttt{config/milla.yaml} \textrightarrow{} \texttt{milla-config.yaml} \textrightarrow{} \texttt{milla.yaml}. A sample is in \texttt{config/milla.yaml}:
\begin{verbatim}
db:
  path: milla_memory.db
node:
  id: home-milla
node:
  id: home-milla
  location: my-lab

ollama:
  url: http://localhost:11434/api/chat
  default_model: llama3.2
  keep_alive: 10m

chat:
  default_session: default
  history_limit: 50

prompt:
  max_tokens: 2000

thermal:
  enabled: false
  max_c: 85
  cooldown_ms: 120000
  sensor_path: /sys/class/thermal/thermal_zone0/temp

server:
  port: 17863
  pid_file: milla.pid
  heartbeat_ms: 5000

log:
  level: info
  file: milla.log
\end{verbatim}
Environment overrides: \texttt{MILLA\_DB\_PATH}, \texttt{MILLA\_NODE\_ID}, \texttt{MILLA\_NODE\_LOCATION}, \texttt{OLLAMA\_URL}, \texttt{OLLAMA\_MODEL}, \texttt{OLLAMA\_KEEP\_ALIVE}, \texttt{CHAT\_HISTORY\_LIMIT}, \texttt{MAX\_PROMPT\_TOKENS}, \texttt{MILLA\_DEFAULT\_SESSION}, or \texttt{MILLA\_CONFIG}.

\section*{Install dependencies}
Dependencies are managed by Clojure CLI via \texttt{deps.edn}; no extra step is needed beyond having \texttt{clj} on PATH. The first run will download Maven deps automatically.

\section*{Initialize the database}
\begin{itemize}
  \item Fast path: \texttt{bin/milla-init-db}
  \item Or: \texttt{clj -M -e "(require 'milla.core) (milla.core/init!)"}
\end{itemize}
This creates \texttt{milla\_memory.db} and required tables (see \texttt{doc/schema.sql}).

\section*{Run the server + prompt}
Start the daemon (recommended for fast responses):
\begin{verbatim}
bin/milla-serve &
# or restart cleanly: bin/milla-restart-server
# stop daemon: bin/milla-stop
# health: bin/milla-health
# migrate schema: bin/milla-migrate
\end{verbatim}

Optional first-run helper:
\begin{verbatim}
bin/milla-setup
\end{verbatim}

All helper scripts accept \texttt{-h}/\texttt{--help}.

Custom helpers can be added in \texttt{src/milla/tools.clj}; you may require JVM/Clojure standard libraries there to build small utilities (keep them pure/safe where possible).

Core entry points to know: \texttt{milla.core/ask!} (builds prompt, sends to Ollama, logs chat), \texttt{milla.core/retry*} (retry wrapper), and \texttt{milla.core/append\_log!} (bounded logging).

\paragraph{Optional CGI chat}
The repo provides \texttt{cgi/milla.cgi} (Perl, requires \texttt{JSON}). It renders a simple chat UI with model dropdown (from \texttt{ollama list}) and buttons to reset the DB/session or download a DB dump (invokes \texttt{bin/milla}, \texttt{bin/milla-reset-db}, \texttt{bin/milla-dump-db} relative to the repo root).
\begin{itemize}[nosep]
  \item Install: copy/symlink \texttt{cgi/milla.cgi} into your web server's CGI path (e.g., \texttt{/usr/lib/cgi-bin}), \texttt{chmod 755} it, and ensure the repo \texttt{bin/} scripts are executable and reachable by the CGI user.
  \item Env: set \texttt{MILLA_CONFIG} for non-default config; ensure the CGI user can read/write the configured DB/log paths. You may set \texttt{MILLA_ROOT} if the repo is outside the CGI working directory. The CGI user must have \texttt{JSON} Perl module installed and permission to run \texttt{ollama list} and the repo \texttt{bin/} scripts.
\end{itemize}

Then send prompts (CLI forwards to the server if running):
\begin{verbatim}
bin/milla llama3.2 "Hello, Milla"
\end{verbatim}
Model argument is optional; if omitted, the configured \texttt{default\_model} is used. For REPL use: \texttt{clj -M:repl} then \texttt{(require 'milla.core)} and call \texttt{(milla.core/ask! {:prompt "Hi"})}.

\section*{Sync (kittens)}
\begin{itemize}
  \item Pull from a remote node: \texttt{bin/milla-sync-db pull user@host:/path/to/milla\_memory.db}
  \item Push to a remote node: \texttt{bin/milla-sync-db push user@host:/path/to/milla\_memory.db}
\end{itemize}
Pulling fetches \texttt{milla\_memory.remote.db} and merges it into the local DB via \texttt{milla.sync/merge} (dedupe by content/time/source).

\section*{Merge two brains}
\begin{verbatim}
bin/milla-merge /path/to/output.db /path/to/db1 /path/to/db2
\end{verbatim}
Unions/dedupes statements/chat/chat\_summaries and reruns summarization per session.

\section*{Self-edit (opt-in, constrained)}
Disabled unless configured with \texttt{:self\_edit \{:enabled true :intent "rag-tools" :allow\_paths [\"src/milla/rag\" \"doc\"] :require\_user\_confirm true\}}. Endpoints (served by \texttt{bin/milla-serve}): \texttt{/code/list}, \texttt{/code/read} (POST JSON \{\texttt{"path": "..."}\}), and \texttt{/code/patch} (POST JSON \{\texttt{"path": "...", "old": "...", "new": "...", "intent": "rag-tools", "confirmed?": true}\}) which performs a guarded replace with a timestamped backup. Requests return 403 when self-edit is disabled or 400 if intent/confirmation is missing. Intended for human-in-the-loop customization of RAG tooling; no automatic execution.

\section*{License}
This project is released under the GNU Affero General Public License v3.0 (AGPL-3.0). See the bundled \texttt{LICENSE} file for full terms.

\section*{Testing}
\begin{verbatim}
clj -M:test
\end{verbatim}
Runs basic smoke tests for init/ask/fact and Ollama retry behavior.

\end{document}
